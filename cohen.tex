\documentclass[10pt,journal,twoside]{IEEEtran}


\usepackage{cite}
\usepackage{amsmath,amssymb,amsfonts}
\usepackage{graphicx}
\usepackage{siunitx}
\usepackage[colorlinks=true,allcolors=blue]{hyperref}
\usepackage{cleveref}
\crefname{equation}{}{}
\Crefname{equation}{}{}
\crefname{figure}{Fig.}{Figs.}
\Crefname{figure}{Fig.}{Figs.}
\crefname{table}{Table}{Tables}
\Crefname{table}{Table}{Tables}
\usepackage{booktabs}
\usepackage{multirow}

\title{An experimental study on electric potential and equipotential regions}
\author{Ryan Cohen\thanks{Author for correspondence: 425rcohen@frhsd.com}, Shreyas Musuku, Justin Hammer, Eshan Handique, Nirvik Patel, Dilan Gandhi, and Nathan Gershteyn\thanks{Authors are with the Science \& Engineering Magnet Program, Manalapan High School, 20 Church Lane, Englishtown, NJ 07726, USA}}
\date{\today}
\markboth{Journal of Science \& Engineering, Vol.~1, No.~2,~December x 2024}{Cohen \MakeLowercase{\textit{et al.}}: An experiental study on equipotential regions}
\setcounter{page}{1}
\newcommand{\keywords}{electric potential, equipotential lines, electric field visualizations, saline conductive medium, electrostatics, electric field mapping, data visualization in electrostatics, gel electrophoresis, electric field dynamics}
\makeatletter
\AtBeginDocument{
\hypersetup{%
pdftitle={\@title},
pdfauthor={\@author},
pdfsubject={physics},
pdfkeywords={\keywords}}}
\makeatother


%\author{\IEEEauthorblockN{Ryan Cohen}
%\IEEEauthorblockA{425rcohen@frhsd.com \\
%\textit{Data Analysis}}
%\and
%\IEEEauthorblockN{Shreyas Musuku}
%\IEEEauthorblockA{425smusuku@frhsd.com \\
%\textit{Conclusion}}
%\and
%\IEEEauthorblockN{Justin Hammer}
%\IEEEauthorblockA{425jhammer@frhsd.com \\
%\textit{Graphs}}
%\and
%\IEEEauthorblockN{Eshan Handique}
%\IEEEauthorblockA{425ehandique@frhsd.com \\
%\textit{Introduction}}
%\and
%\IEEEauthorblockN{Nirvik Patel}
%\IEEEauthorblockA{425npatel@frhsd.com \\
%\textit{Graphs}}
%\and
%\IEEEauthorblockN{Dilan Gandhi}
%\IEEEauthorblockA{525dgandhi@frhsd.com \\
%\textit{Abstract \& Analysis}}
%\and
%\IEEEauthorblockN{Nathan Gershyten}
%\IEEEauthorblockA{425ngershteyn@frhsd.com \\
%\textit{Procedure}}
%}





\begin{document}
\maketitle

\begin{abstract}
The goal of this experiment was to observe and analyze the distribution of electric potential around conducting wires and to locate the corresponding equipotential regions. We were able to achieve this by using salt water as a conductive medium, we then measured the potential difference at various set points by adjusting the voltage at each measurement location. Our observations revealed that the equipotential regions in the center were equidistant from both voltage sources, and the potential change along each parallel path, at equal distances from its nearest voltage source, was consistent. While some minor discrepancies between the expected and actual results can be attributed to human error, the experiment effectively demonstrated the properties of electric potential and provided a reliable confirmation of the relationship between electric potential and equipotential lines.
\end{abstract}

\begin{IEEEkeywords}
\keywords
\end{IEEEkeywords}

\section{Introduction}
An electric field is a field created by the attraction and repulsion of the charges that inhabit the field. A force is exerted on everything within the field, and the strength of the charges and the distance between them decides the strength of this force. Electric potential is the difference in potential energy between two places in an electric field. Electric potential ($V$) can be derived from the electric field ($\vec{E}$) using the following equation that is wacky and wrong:
\begin{equation}
V = -E \, dr
\end{equation}
Equipotential lines are lines with the same electric potential along the line. As seen in the aforementioned equation, electric field and electric potential point in opposite directions, as indicated with the negative sign. Along with the fact that the electric field and equipotential lines are perpendicular to each other, this means that the electric field always points out towards equipotential lines of lesser electric potential.

\section{Methods and materials}
In this experiment, we began by gathering a water tray, a \qty{12}{\volt} power supply, a multimeter, two insulated wires, and graph paper. To increase conductivity, we added a small amount of salt to the water, which helped create a more uniform electric field throughout the tray. We then positioned the graph paper beneath the tray to establish a grid, allowing us to measure and record voltage at precise coordinates. Next, we set up our circuit by connecting one insulated wire to the positive terminal of the power supply and submerging it at one end of the water tray, while the other wire, connected to the negative terminal, was placed at the opposite end. This setup generated an electric field across the tray. With our equipment ready, we set the multimeter to measure voltage and began by positioning its probe at a reference ground \qty{0}{\volt} point. From there, we measured the voltage at each point on the grid, moving outward from the ground point to cover various coordinates across the tray. We recorded the voltage readings at each coordinate, adjusting the power supply incrementally from \qtylist{3;6;9;12}{\volt} to observe how the potential changed with different voltages. We used the recorded data to create a 2-D map illustrating the equipotential regions within the tray. Finally, we analyzed the data to observe where the rate of potential change was highest compared to the flatter equipotential regions that formed as we moved farther from the source. This allowed us to identify specific equipotential regions where the potential remained almost constant, also indicating areas where the electric field does not change.

\begin{figure}
\centering
\begin{minipage}{0.50\columnwidth}
    \includegraphics[width=\linewidth]{Fig1.png}
    \caption{Image of Lab Setup}
    \label{fig:setup}
\end{minipage}\hfill
\begin{minipage}{0.50\columnwidth}
    \includegraphics[width=\linewidth]{Fig2.png}
    \caption{Diagram of Setup}
    \label{fig:diagram}
\end{minipage}
\end{figure}



\section{Resutlts}
\begin{figure}
    \centering
    \includegraphics[width=0.9\linewidth]{Fig3.png}
    \caption{Recorded Voltage at Different Points}
    \label{fig:voltage_data}
\end{figure}

The data collected at voltage levels of \qtylist{3;6;9;12}{\volt} showed a consistent increase in potential with distance from the ground point, reflecting how potential rises near the source and weakens as it spreads outward. For example, at \qty{3}{\volt}, the potential increased from \qty{0.53}{\volt} at Point 1 to \qty{1.48}{\volt}Vat Point 9. At higher voltage levels, such as \qty{9}{\volt}, potential rose from \qty{3.47}{\volt} at Point 1 to \qty{7.98}{\volt} at Point 9, although some outliers such as \qty{15.2}{\volt} at Point 3 suggest that there were field variations or that the data was misread.

Using this data, we constructed a 2-D map of equipotential regions, where areas with similar potential are represented as color-coded bands. In the map, potential changes most rapidly near the electrodes, where the electric field is strongest, shown by tightly packed color bands. As the distance from the source increases, the potential changes more gradually, creating broader equipotential regions that are farther apart, reflecting a weakening electric field. Adding salt to the water enhanced conductivity, creating a stable field environment for more consistent measurements. 

\begin{figure}
    \centering
    \includegraphics[width=1\linewidth]{Fig4.png}
    \caption{Equipotential Map}
    \label{fig:equipotential_map}
\end{figure}

\section{Discussion}
The stronger electric field near the electrodes is marked by steeper potential gradients, while farther regions, with slower potential shifts, indicate weaker fields. The inferred electric field lines would radiate outward from the positive terminal and converge at the negative, crossing equipotential surfaces perpendicularly. This pattern aligns with the theoretical relationship between electric fields and equipotentials, demonstrating how electric field strength correlates with the rate of potential change in a conductive medium.

\section{Conclusion}
The experiment effectively demonstrated the idea of equipotential lines. The electric potential along a theoretical equipotential line stayed somewhat constant, proving how electric potential is the same along an equipotential line. Even with this, the experiment could have been improved. The voltmeter at some points seemed to not be very precise, and the power supply may have been supplying a slightly incorrect voltage due to wear and tear. In addition, human error would have caused some incorrect readings with the voltage. If this experiment were to be conducted again, more modern technology could be used, along with more precise measuring.

\section{Acknowledgement}
We thank several anonymous reviewers whose comments helped our manuscript.  RC did data analysis. SM wrote the conclusions. JH plotted the results. EH wrote the introduction. NP plotted the results. DG analyzed results. NG wrote the methods and materials.  

\nocite{tipler}
\bibliographystyle{IEEEtran}
\bibliography{lab.bib}

\begin{IEEEbiography}[{\includegraphics[width=1in,height=1.25in,clip,keepaspectratio]{rcohen.jpeg}}]{Ryan Cohen} is a senior in the Science and Engineering Magnet Program at Manalapan High School and has completely redesigned NYC Pier 6. 
\end{IEEEbiography}
\begin{IEEEbiography}[{\includegraphics[width=1in,height=1.25in,clip,keepaspectratio]{smusuku.jpeg}}]{Shreyas Musuku} is a senior in the Science and Engineering Magnet Program at Manalapan High School and enjoys the magical world of wearable computers for first responders. 
\end{IEEEbiography}
\begin{IEEEbiography}[{\includegraphics[width=1in,height=1.25in,clip,keepaspectratio]{jhammer.jpeg}}]{Justin Hammer} is a senior in the Science and Engineering Magnet Program at Manalapan High School and enjoys the magical world of aneurysms. 
\end{IEEEbiography}
\begin{IEEEbiography}[{\includegraphics[width=1in,height=1.25in,clip,keepaspectratio]{ehandique.jpeg}}]{Eshan Handique} is a senior in the Science and Engineering Magnet Program at Manalapan High School and enjoys the magical world of aneurysms. 
\end{IEEEbiography}
\begin{IEEEbiography}[{\includegraphics[width=1in,height=1.25in,clip,keepaspectratio]{npatel.jpeg}}]{Nirvik Patel} is a senior in the Science and Engineering Magnet Program at Manalapan High School and enjoys the magical world of using news and events to predict currency markets. 
\end{IEEEbiography}
\vfill\newpage
\begin{IEEEbiography}[{\includegraphics[width=1in,height=1.25in,clip,keepaspectratio]{dgandhi.jpeg}}]{Dilan Gandhi} is a senior in the Science and Engineering Magnet Program at Manalapan High School and must unlearn what he has learned. 
\end{IEEEbiography}
\begin{IEEEbiography}[{\includegraphics[width=1in,height=1.25in,clip,keepaspectratio]{ngershteyn.jpeg}}]{Nathan Gershteyn} is a senior in the Science and Engineering Magnet Program at Manalapan High School and enjoys the magical word of using news and events to predict currency markets. 
\end{IEEEbiography}
\vfill
\end{document}