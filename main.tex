\documentclass[conference]{IEEEtran}
\IEEEoverridecommandlockouts
\usepackage{cite}
\usepackage{amsmath,amssymb,amsfonts}
\usepackage{graphicx}
\usepackage{float}
\usepackage{textcomp}
\usepackage{xcolor}
\def\BibTeX{{\rm B\kern-.05em{\sc i\kern-.025em b}\kern-.08em
    T\kern-.1667em\lower.7ex\hbox{E}\kern-.125emX}}
    
\begin{document}

\title{Visualizing Electric Potential: Mapping Equipotential Lines in a Conductive Water Tray}

\author{
\IEEEauthorblockN{Ryan Cohen}
\IEEEauthorblockA{425rcohen@frhsd.com \\
\textit{Data Collection \\ Discussion}}
\and
\IEEEauthorblockN{Eshan Handique}
\IEEEauthorblockA{425ehandique@frhsd.com \\
\textit{Data Collection \\ Introduction}}
\and
\IEEEauthorblockN{Dilan Gandhi}
\IEEEauthorblockA{525dgandhi@frhsd.com \\
\textit{Data Collection \\ Abstract}}
\and
\IEEEauthorblockN{Shreyas Musuku}
\IEEEauthorblockA{425smusuku@frhsd.com \\
\textit{Data Collection \\ Conclusion}}
\and
\IEEEauthorblockN{Nathan Gershteyn}
\IEEEauthorblockA{425ngershteyn@frhsd.com \\
\textit{Data Collection \\ Procedure}}
\and
\IEEEauthorblockN{Nirvik Patel}
\IEEEauthorblockA{425npatel@frhsd.com \\
\textit{Data Collection \\ Graphs/Setup}}
\and
\IEEEauthorblockN{Justin Hammer}
\IEEEauthorblockA{425jhammer@frhsd.com \\
\textit{Data Collection \\ Graphs}}
}

\maketitle

\begin{abstract}
The goal of this experiment was to study the distribution of electric potential around conducting wire configurations and identify the corresponding equipotential regions. Using a water tray with a saline solution as the conductive medium, we measured the potential difference at defined points on a coordinate grid by connecting the wires to a power supply and using a multimeter to record voltages at various locations. Our observations indicated that equipotential regions were centered somewhat equidistantly between the voltage sources, with a mostly consistent rate of potential change along paths parallel to the sources. Some discrepancies between expected and observed values were noted, likely due to limitations in the measurement equipment and procedural inconsistencies. Despite these challenges, the experiment successfully demonstrated the properties of electric potential and equipotential lines, providing a clear visual representation of their relationship with the electric field.
\end{abstract}

\begin{IEEEkeywords}
Electric Potential, Equipotential Lines, Electric Fields, Saline Solution, Conductive Medium, Voltage Mapping, Potential Gradient
\end{IEEEkeywords}

\section{Introduction}
Electric fields arise from the presence of electric charges and describe the force that a charge would experience within the region of influence. The strength of the electric field is determined by the magnitude and distribution of charges, as well as the distance from the source. Electric potential, on the other hand, represents the potential energy per unit charge at a point in the field and is related to the electric field through the equation:
\begin{equation}
V = -E \, dr
\end{equation}
This equation shows that electric potential is the integral of the electric field along a path, with the negative sign indicating that electric potential decreases in the direction of the field. Equipotential lines are defined as contours where the electric potential remains constant. These lines are always perpendicular to electric field lines, reflecting the fact that no work is required to move a charge along an equipotential surface. This property makes equipotential mapping a useful tool for visualizing the distribution of electric fields and potential in various setups.

In this experiment, we used a water tray filled with a saline solution as a conductive medium to simulate the behavior of electric potential and field in a simplified, accessible environment. Salt was added to enhance the conductivity of the water, ensuring a stable field for measurement. By placing conducting wires connected to a power supply in the tray, we created an electric field and measured the potential at various points using a multimeter. This allowed us to map the equipotential regions and explore their relationship with the electric field. The results gave us insights into how equipotential lines and their relationship to electric fields are critical in applications such as designing capacitors, grounding systems, and electrical shielding.

\section{Procedure}
To study electric potential and equipotential regions, we used the following materials: a water tray, a 12V DC power supply (model PS-28A), a digital voltmeter (model HSPE-051 AC), two insulated copper wires, graph paper, and a small amount of table salt (about 1/16 teaspoon or a pinch). The salt was dissolved in the water to enhance its conductivity, creating a stable conductive medium for electric field visualization. The water tray dimensions were approximately 15cm by 15cm, ensuring sufficient space for accurate mapping of the potential.

We began by securing graph paper beneath the tray to establish a grid system for voltage measurements, such that each grid square corresponded to 1cm by 1cm. A small amount of blue food coloring (about 5 drops) was added to the water to improve visibility during the setup and ensure uniform distribution of the solution.

The circuit was set up by attaching one wire to the positive terminal of the power supply and submerging it at one end of the tray, while the other wire was connected to the negative terminal and submerged at the opposite end. This configuration created an electric field across the tray. We placed the wires parallel to the graph paper's grid lines for consistency in measurements. Points of measurement (labeled 1 through 9) were marked at specific coordinates on the grid, spaced uniformly to facilitate data collection.

Using a multimeter set to measure voltage, we started at a reference ground point (0V) and systematically recorded the voltage at each grid intersection, moving outward from the ground point. Measurements were taken at incremental voltage settings of 3V, 6V, 9V, and 12V, adjusted at the power supply. This allowed us to observe how potential varied across the tray for different input voltages.

Data collected at these points were used to construct a 2D map of the tray, highlighting equipotential regions as distinct bands. The areas of rapid potential change (near the electrodes) and flatter equipotential regions (farther from the electrodes) were analyzed. These observations were consistent with the expected behavior of the electric field and potential distribution in a conductive medium.

\begin{figure}[ht]
    \centering
    \begin{minipage}{0.45\textwidth}
        \includegraphics[width=\linewidth]{image3.png}
        \caption{Image of Lab Setup}
        \label{fig:setup}
    \end{minipage}
\end{figure}

\begin{figure}[ht]
    \centering
    \begin{minipage}{0.45\textwidth}
        \includegraphics[width=\linewidth]{image2.png}
        \caption{Diagram of Setup}
        \label{fig:diagram}
    \end{minipage}
\end{figure}

\section{Results}

The data collected at input voltage levels of 3V, 6V, 9V, and 12V consistently showed an increase in potential as the distance from the ground point increased. For example, at 3V, the potential ranged from 0.53V at Point 1 to 1.48V at Point 9. At 9V, the potential rose from 3.47V at Point 1 to 7.98V at Point 9. However, there were outliers, such as a value of 15.2V at Point 3. This suggests either variations in field uniformity or higher concentrations of salinity in that region.

Using this data, we generated a detailed equipotential map using Python’s matplotlib library (see Fig. \ref{fig:equipotential_map}), which visually represents the regions of equal electric potential across the water tray. Equipotential lines are more tightly packed near the electrodes, indicating steep potential gradients and stronger electric fields. As the distance from the electrodes increases, the equipotential lines become more widely spaced, reflecting weaker electric fields and more gradual potential changes.

\begin{figure}[ht]
    \centering
    \includegraphics[width=1\linewidth]{image.png}
    \caption{Recorded Voltage at Different Points}
    \label{fig:voltage_data}

        \vspace{5mm} 
    \includegraphics[width=1\linewidth]{equimapge.png}
    \caption{Equipotential Map}
    \label{fig:equipotential_map}
\end{figure}



 \newpage \section{Discussion}
The potential changes most rapidly near the electrodes, where the electric field strength is highest. This is evident in the tightly packed equipotential lines close to the voltage sources in Fig. \ref{fig:equipotential_map}. Farther from the electrodes, the potential change becomes more gradual, as reflected in the broader spacing between lines. This indicates a diminishing electric field strength as distance from the source increases.

Salt was added to the water to increase its conductivity, ensuring a uniform and stable conductive medium. The salt ions facilitated current flow, allowing the electric field to propagate more effectively and improving the consistency of voltage measurements across the tray.

The electric field radiates outward from the positive terminal and converges at the negative terminal. The regions with steeper potential gradients correspond to stronger electric fields, while flatter potential regions indicate weaker fields. This relationship reflects the fundamental correlation between electric fields and the rate of change of potential.

\section{Conclusion}
The experiment effectively demonstrated the idea of equipotential lines. The electric potential along a theoretical equipotential line stayed somewhat constant, proving how electric potential is the same along an equipotential line. Even with this, the experiment could have been improved. The voltmeter at some points seemed to not be very precise, and the power supply may have been supplying a slightly incorrect voltage due to wear and tear. In addition, human error in terms of correct probe placement would have caused some incorrect readings with the voltage. If this experiment were to be conducted again, more modern technology could be used, along with more precise measuring.

        \vspace{90mm} 

\section*{Contributions}
\begin{itemize}
    \item \textbf{Ryan Cohen:} Data collection, Discussion
    \item \textbf{Eshan Handique:} Data collection, Introduction
    \item \textbf{Dilan Gandhi:} Data collection, Abstract
    \item \textbf{Shreyas Musuku:} Data collection, Conclusion
    \item \textbf{Nathan Gershteyn:} Data collection, Procedure
    \item \textbf{Nirvik Patel:} Data collection, Graphs/Setup
    \item \textbf{Justin Hammer:} Data collection, Graphs
\end{itemize}

\section*{References}
\begin{enumerate}
    \item "Equipotential Lines," HyperPhysics, Georgia State University. [Online]. Available: \texttt{http://hyperphysics.phy-astr.gsu.edu/hbase/electric/equipot.html}. [Accessed: Dec. 2, 2024].
    \item "Equipotential Surfaces and Lines," Physics LibreTexts. [Online]. Available: \texttt{https://phys.libretexts.org/Bookshelves/University\_Physics/Physics\_(Boundless)/18\%3A\_Electric\_Potential\_and\_Electric\_Field/18.2\%3A\_Equipotential\_Surfaces\_and\_Lines}. [Accessed: Dec. 2, 2024].
    \item "Matplotlib Introduction," W3Schools. [Online]. Available: \texttt{https://www.w3schools.com/python/matplotlib\_intro.asp}. [Accessed: Dec. 2, 2024].
\end{enumerate}

\end{document}